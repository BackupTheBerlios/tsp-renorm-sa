\begin{abstract}
We propose a new strategy to solve the euclidean TSP. In the euclidean TSP the goal is to acquire the
shortest tour which visits all the cities specified by a point in the plane. The distance is calculated 
using the euclidean distance. The euclidean TSP is solved using the deterministic renormalization 
method. In renormalization at first a coarse grained route is estimated. This is done by dividing the area
into four cells. The exact shortest route is determined, which connects the cells where at least one city is 
located. This route is refined by recursively dividing each cell and determine the shortest route again. 
If this process is continued until 
each cell contains at most one city, the estimated tour is known. The only problem is how the grid is 
positioned. There is analyzed if thermodynamic simulated annealing can optimize the results found by 
renormalization. In thermodynamic simulated annealing the TSP system is considered as a metal which 
is cooling down. The energy in the system is equal to the length of the tour. In a hot state the rotation can 
be changed significantly. If the metal is becoming colder the allowed rotations on the grid decreases.
This allows to find optimal solutions without getting stuck in a local optima. Our results demonstrate that there
is an improvement of 21 percent over the worst rotation angle. As the number of cities increase, the accuracy
of the estimation decreases only 0.0180 percent per city.  
\end{abstract}