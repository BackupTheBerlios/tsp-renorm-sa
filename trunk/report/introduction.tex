\section{Introduction}
The Traveling Salesman Problem (TSP) is a very well known problem. The problem
deals with a traveling salesman which needs to visit a set of cities. Since
time is money, he would like to take the shortest route possible, but still
visit all cities. This is a simple definition, but TSP is actually a quite
large set of combinatorial problems. Examples types of  TSP problems are:

\begin{itemize}
\item \textsc{Symmetric TSP:}  In this problem there is given a complete graph.
The nodes of these graph are the cities and all cities are connected by
undirectional edges. The weight of the edge is the distance between the cities.
The task is now to find the shortest tour which visits all cities in the graph,
using each edge at most once (Hamiltonian tour). This tour has to end at the
starting city.
\item \textsc{Euclidean TSP:}  Euclidean TSP is a subset of the Symmetric TSP.
Here all the cities are specified by a point in the plane. The weight of the
edges is calculated using the euclidean distance. It is now again the task to
find the shortest hamiltonian tour.
\item \textsc{Asymmetric TSP:} This TSP problem is related to the Symmetric
TSP. In this case we do not have a complete graph and the edges between the
cities are directed.
\item \textsc{Chinese Postmen Problem:} With the Chinese Postmen problem,  the
requirement to find a Hamiltonian tour is somewhat relaxed. It is now allowed
to visit an edge more than once.
\end{itemize}

The interesting aspect of the TSP problem can be found in theoretical and
practical aspects. Theoretically the TSP problem is NP-hard. A NP-hard
algorithm is a problem which can not easily be solved. In order to solve the
problem you can not do better than simply try all possible solutions. These
possible solutions can be checked in polynomial time.  One can prevent checking
all possible solutions by using an algorithm which gives an approximation of
the best solution.

In practice, the TSP problem is widely used for example in designing  VLSI.
VLSI is an abbreviation of very large scale integration where a very large
number of components needs to be integrated on a single chip.

Many research fields focus their attention on the Traveling Salesman Problem.
Disciplines as Mathematics, Physics, Biology, Artificial Intelligence and
Computer Science. Each field provides their own unique insight into the
problem. In this paper we will investigate some interesting approaches from
physics. Here two methods have drawn our attention. These approaches are Real
Space Renormalization Theory and Thermodynamic Simulated Annealing. 

This paper is structured as follows. The second section will explain
Renormalization Theory. The third section will describe Thermodynamic Simulated
Annealing and how this is used in combination with the Renormalization Theory.
The fourth section will treat the results of our method. Finally we will
conclude with a short conclusion.
% vim:ft=tex:spell spelllang=en:autoindent


