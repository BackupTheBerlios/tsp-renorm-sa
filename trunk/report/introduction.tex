\section{Introduction}
\IEEEPARstart{T}{he} Traveling Sales Person (TSP) problem\cite{lcns1994} is a very well known
problem first proposed around 1800 by Sir William Rowan Hamilton and Thomas
Penyngton Kirkman. The problem deals with a traveling salesman which needs to
visit a number of cities. Since time is money, he would like to take the
shortest route possible, but still visit all cities. This is a simple
definition, but TSP is actually a quite large set of combinatorial problems.
Examples of TSP problems are:
\begin{itemize}
\item \textsc{Symmetric TSP problem:}  In this problem there is given a complete graph.
The nodes of these graph are the cities and all cities are connected by
unidirectional edges. The weight of the edge is the distance between the cities.
The task is now to find the shortest tour which visits all cities in the graph,
using each edge at most once (Hamiltonian tour). This tour has to end at the
starting city.
\item \textsc{Euclidean TSP:}  Euclidean TSP problem is a subset of the
Symmetric TSP problem.  Here all the cities are specified by a point in the
plane. The weight of the edges is calculated using the euclidean distance. It
is now again the task to find the shortest Hamiltonian tour.
\item \textsc{Asymmetric TSP problem:} This TSP problem is related to the
Symmetric TSP. In this case we do not have a complete graph and the edges
between the cities are directed.
\item \textsc{Chinese Postmen Problem:} With the Chinese Postmen problem,  the
requirement to find a Hamiltonian tour is somewhat relaxed. It is now allowed
to visit an edge more than once.
\end{itemize}

The TSP problem is interesting from a theoretical aspect as well as from a
practical aspect. Theoretically the TSP problem is NP-hard i.e. it can not be
solved easily. In order to find the optimal solution for the problem all the
possible solutions have to be analyzed. Which can be done in polynomial time.
Better approaches exist to solving a TSP problem, however, the algorithms
which can be used only do an approximation.

Due to the quantity of problems which resemble the TSP problem many attempts
have been made to find the optimal solution. Algorithms which mimic processes
in various disciplines have been used. Examples of such processes are ant
colonies, genetic mutation, neural networks, etc. 

Real space renormalization\cite{yoshiyuki1995nms}, a technique from the field of theoretical physics,
has been put forward as a fast approximation technique of the shortest route.
However, this technique relies heavily on the distribution of the cities in
the TSP problem. In this paper we analyze if simulated annealing, a process
which has its analogy from the field of metallurgy, can improve the results
found by renormalization.

This paper is structured as follows. The next section delves deeper in the two
algorithms and motivates the combination. The third section treats the results
found. The last section concludes this work.
\IEEEpubidadjcol % needed for the copyright notice
% vim:ft=tex:spell spelllang=en:autoindent


