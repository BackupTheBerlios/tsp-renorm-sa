\documentclass[twocolumn]{article}
\usepackage{graphicx}
\usepackage{amssymb,amsmath}
\author{A. Bossenbroek P. van Stralen\\University of Amsterdam\\\{abossenb, pstralen\} AT science.uva.nl}
\title{Solving the Euclidean TSP Problem using Simulated Annealing and Renormalization }
\begin{document}
\maketitle
\section{Introduction}
\section{Renormalization Theory}
Renormalization Theory is a deterministic approach to solving the TSP problem. The renormalization theory is originating from theoretical physics. In theoretical physics renormalization theory is used for investigating the changes of a physical system, by viewing it at different scales.
\newline\newline\noindent
When we are applying renormalization theory to the TSP problem, we first solve the TSP at a large scale. This solution is used recursively to solve it a smaller scale. This is repeated until the scale is small enough for retrieving a hamiltonian path connecting all cities.
\newline\newline\noindent
Now we summerarized the method, we go more into detail. The solving process is started by calculating the shortest path for a two by two block
\section{Thermodynamic Simulated Annealing}
\section{Results}
\section{Conclusion}
\section{References}
\end{document}
